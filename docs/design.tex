\documentclass{utart}

\utUseMinted
\usepackage{outlines}

\title{DIM 设计文档}
\author{阙知勇}

\setUTClassify{绝密}

% 设置文档编号
\setUTIndex{UT-A0001}

% 设置拟制人信息
\setUTFiction{lxz}{2022-03-22}

% 设置审核人信息
\setUTReview{lxz}{2022-03-22}

% 设置批准人信息
\setUTApprove{lxz}{2022-03-22}

\begin{document}

% 生成封面
\utMakeTitle{}

% 生成修订记录
\utMakeChangeLog{%
    1.0.0 &      创建   &    李鹤   &   2020-02-20      \\
    \hline
}

% 生成目录
\utMakeTOC

% 文档内容
\section{概述}

概述。

\section{系统设计}

    \subsection{设计原则}

        设计原则。

    \subsection{模块结构设计}
        \begin{plantuml}
        @startuml
        package "DIM core" {
            class Dim {
                + postEvent()
                - postInputContextCreated()
                - postInputContextDestroyed()
                - postInputContextFocused()
                - postInputContextUnfocused()
                - postInputContextKeyEvent()
            }
            class Addon {
                + string key()
            }
            class FrontendAddon extends Addon {
            }
            class InputMethodAddon extends Addon {
                + {abstract} list<InputMethodEntry> getInputMethods()
                + {abstract} keyEvent(InputMethodEntry, InputContextKeyEvent)
            }
            class InputState {
            }
            class InputContext {
                InputState inputState
            }
        }

        package "DBus Frontend" {
            class DBusFrontend extends FrontendAddon {
                + CreateInputContext()
            }

            class InputContext1 extends InputContext {
                + Destroy()
                + FocusIn()
                + FocusOut()
                + ProcessKeyEvent()
            }
        }

        package "keybaord" {
            class Keyboard extends InputMethodAddon {
                + GetInputMethods()
                + keyEvent()
            }
        }

        package "Fcitx5proxy" {
            class Fcitx5proxyAddon extends InputMethodAddon {
                + GetInputMethods()
                + keyEvent()
            }
        }

        DBusFrontend::CreateInputContext --> InputContext1
        InputContext::inputState --> InputState

        @enduml
        \end{plantuml}

        \pagebreak
        \par 时序图(以使用 Qt 输入法插件来与 DIM 交互的 Qt 程序为例):

        \begin{plantuml}
        @startuml

        actor User
        participant "程序" as Process
        box Addons #LightBlue
        participant "InputContext1"
        participant "DBus Frontend" as DF
        participant "InputMethod Addon" as IMA
        participant "UI Addon" as UA
        endbox
        participant "DIM core" as Dimcore

        autoactivate on
        Dimcore -> UA ** : new
        Dimcore -> DF ** : new
        Dimcore -> IMA ** : new
        Dimcore -> IMA : getInputMethods
        return 返回插件提供的输入法列表
        User -> Process ** : 打开
        Process -> DF : CreateInputContext
        DF -> InputContext1 ** : new InputContext1
        DF -> Dimcore : 触发 InputContextCreated 信号
        return
        return 返回 InputContext ID

        alt 程序得到焦点
            User -> Process : 切换到应用
                Process -> InputContext1 : FocusIn
                    InputContext1 -> Dimcore : 触发 InputContextFocused 信号
                        Dimcore -> Dimcore : 设置当前输入上下文 ID
                        return
                    return
                return
            return
        else 程序失去焦点
            User -> Process : 切出应用
                Process -> InputContext1 : FocusOut
                    InputContext1 -> Dimcore : 触发 InputContextUnfocused 信号
                        Dimcore -> Dimcore : 将当前输入上下文 ID 设置为 0
                        return
                    return
                return
            return
        else 用户按键
            User -> Process : 按键
                Process -> InputContext1 : ProcessKeyEvent
                    InputContext1 -> Dimcore : 触发 InputContextKeyEvent 信号
                        Dimcore -> Dimcore : 检查是不是当前 Context 发出的信号
                        return
                        alt 是快捷键
                            Dimcore -> Dimcore : 处理快捷键
                            return
                            Dimcore -> UA : 设置提示内容
                            return
                        else 不是快捷键
                            Dimcore -> IMA : keyEvent
                                IMA -> IMA : 处理按键内容
                                return
                                IMA -> Dimcore : 触发 updatePreedit、commitString、updateLookupTable 信号
                                    Dimcore -> UA : 更新显示内容
                                    return
                                return
                            return
                        end
                    return
                return
                Process -> Process : 解析返回结果,\n处理 commitString、\nupdatePreedit、\nforwardKey
                return
                Process -> Process : 获取光标位置
                    Process -> DF : SetCursorRect
                        DF -> Dimcore : InputContextCursorRectChanged
                            Dimcore -> UA : 设置光标位置
                            return
                        return
                    return
                return
            return
        end

        @enduml
        \end{plantuml}

        \subsubsection{模块列表}
        \begin{outline}[enumerate]
            \1 DIM core:核心模块,负责处理输入法的核心逻辑,包括输入法的初始化、配置读取、插件管理、事件处理及分发、状态管理。
            \1 插件:各种功能由插件来实现,DIM core 通过插件接口与插件交互。
                \2 前端插件
                \2 输入法插件
                    \3 键盘布局插件:获取系统支持的键盘布局,当切换到键盘布局插件下的布局时,按键事件会被分发到此插件。插件获取当前布局的键盘映射表,并将对应的映射返回。
                    \3 Fcitx5 代理插件:用于兼容 Fcitx5 的输入法,调用 Fcitx5 的接口获取输入法列表。当切换到插件下的输入法时,按键事件会被分发到此插件。插件将按键事件转发给 Fcitx5,并获取结果返回。
                    \3 iBus 代理插件:用于兼容 iBus 的输入法,调用 iBus 的接口获取输入法列表。当切换到插件下的输入法时,按键事件会被分发到此插件。插件将按键事件转发给 iBus,并获取结果返回。
                \2 前端插件:支持用于与程序交互的前端插件,包括 DBus 前端、XIM 前端、Wayland IM 前端,其中 DBus 前端同时供 Qt 与 GTK 程序使用。
                    \3 DBus 前端插件:用于通过 DBus 与需要输入的程序交互,包括 Qt 程序与 GTK 程序。DBus 前端会监听 DBus 的消息,当收到消息时,将触发事件给 DIM core,DIM core 处理完后,将结果返回给 DBus 前端,DBus 前端再将结果返回给输入的程序。
                    \3 XIM 前端插件:实现 X 输入法协议前端
                    \3 Wayland 前端插件:实现支持 Wayland 的 inputmethod 协议的前端插件。
                \3 UI 插件:用于显示输入法候选词、状态等
                    \4 输入面板插件
                    \4 KIM panel 插件
            \1 配置文件模块
            \1 Qt 输入法支持:在 Qt 程序中使用 DIM,通过 DBus 前端与 DIM 交互。
            \1 GTK 输入法支持:在 GTK 程序中使用 DIM,通过 DBus 前端与 DIM 交互。
            \1 im-config 配置文件:增加 im-config 的配置文件,在系统安装 DIM 后,自动选择 DIM 并导出环境变量。
        \end{outline}

        \subsubsection{事件类型}
        \begin{outline}[enumerate]
            \1 InputContextCreated,输入上下文创建事件,由前端插件触发。
            \1 InputContextDestroyed,输入上下文销毁事件,由前端插件触发。
            \1 InputContextFocused,输入上下文得到焦点,由前端插件触发。
            \1 InputContextUnfocusd,输入上下文失去焦点,由前端插件触发。
            \1 InputContextKeyEvent,按键事件,由前端插件触发,事件触发后,DIM core 先处理掉切换输入法等快捷键的事件,剩下的事件分发给当前选择的输入法插件。
        \end{outline}

    \subsection{接口设计}

        \subsubsection{inputmethod 接口}
            \par inputmethod 接口用于给程序提供输入法操作功能,包括创建输入上下文。
            \paragraph{方法}
                \begin{outline}[enumerate]
                    \1 CreateInputContext 用于创建输入上下文,当程序需要输入时,需要调用此方法创建输入上下文。
                \end{outline}

        \subsubsection{inputcontext 接口}
            \par inputcontext 接口用于操作上下文,包括处理按键事件、获取当前输入法的状态、获取当前输入法的候选词等。
            \paragraph{方法}
                \begin{outline}[enumerate]
                    \1 Destroy 用于销毁上下文,当窗口关闭时,需要调用此方法销毁上下文。
                    \1 FocusIn 用于在输入窗口获取焦点时,调用方法。
                    \1 FocusOut 用于在输入窗口失去焦点时,调用方法。
                    \1 ProcessKeyEvent 处理按键事件,并返回处理结果。签名:\mintinline[breaklines]{c++}{ProcessKeyEvent(uint keyval, uint keycode, uint state, bool isRelease, uint time) => []variant}
                        \par 返回结果示例:
                        \begin{minted}{c++}
                            // 1
                            [
                                {
                                    type: 1, // preedit string
                                    data: "ce ui",
                                },
                            ]
                            // 2
                            [
                                {
                                    type: 1, // preedit string
                                    data: "",
                                },
                                {
                                    type: 2, // commit string
                                    data: "测试",
                                },
                            ]
                        \end{minted}
                \end{outline}

% 结束文档
\end{document}
